
\usepackage{fancyhdr}
\usepackage{lastpage}
\usepackage[utf8]{inputenc}

% Minted for syntax highliting
\usepackage{minted}
\usemintedstyle{tango}

% Headers/footers styling
\pagestyle{fancy}
\fancyhf{}
\renewcommand{\headrulewidth}{0pt}

% Footer
\lfoot{ID1019}
\cfoot{KTH}
\rfoot{\thepage \hspace{1pt} / \pageref{LastPage}}

%\newcommand{\defaultpagestyle}{\thispagestyle{plain}}
\newcommand{\defaultpagestyle}{\thispagestyle{fancy}}


\title[ID1019 Maps and Structs]{Maps and Structs}


\author{Johan Montelius}
\institute{KTH}
\date{\semester}

\begin{document}

\begin{frame}
\titlepage
\end{frame}


\begin{frame}[fragile]{Java vs Elixir}

  \pause
  \begin{columns}
    \begin{column}{0.6\textwidth}
\begin{verbatim}
public static int fib(int n) {

    if ( n == 0 )
      return 0;

    int n2 = 0, n1 = 1;
    int ni = n1;

    for(int i = 1; i < n; i++) { 
        ni = n1 + n2;  
        n2 = n1;
        n1 = ni; 
     }
     return ni;
};
\end{verbatim}      
    \end{column}
    \begin{column}{0.4\textwidth}
\begin{verbatim}
def fib(0) do  0 end
def fib(1) do  1 end
def fib(n) do
  fib(n-1) + fib(n-2)
end
\end{verbatim}
    \end{column}
  \end{columns}


\end{frame} 

\begin{frame}[fragile]{Elixir type specification}

\pause\vspace{10pt}
\begin{verbatim}
 @spec fib(integer()) :: integer()

 def fib(0) do  0 end
 def fib(1) do  1 end
 def fib(n) do
   fib(n-1) + fib(n-2)
 end
\end{verbatim}

\vspace{20pt}\pause The compiler does not care about type specifications!

\vspace{10pt}\pause Compiles ok:
\begin{verbatim}
        :
     fib(:bananas)
        :
\end{verbatim}

\end{frame}



\begin{frame}{types in Elixir}

What types do we have:

\pause\vspace{10pt}
Singletons, the types of individual data structures:
\begin{itemize}
\item  1, 2 or 42
\item :foo, :bar or :atom
\item  \{:foo, 42\}
\end{itemize}
\pause\vspace{10pt}
Unions of singletons, what we normally refer to as ``types'':
\begin{itemize}
\item integer(): any integer value 
\item float(): any floating point value 
\item atom(): any atom 
\item pid(): a process identifier
\item reference(): a reference
\item fun(): a function 
\item .. and many more
\end{itemize}

{\em Could also be written without the ``()''. }
\end{frame}

\begin{frame}{types in Elixir}
Types for compound data structures:
\pause\vspace{10pt}
\begin{itemize}
\item  tuples: {\tt \{\}}, {\tt \{atom(), integer()\}}, ....
\item  lists: {\tt [integer()]}, {\tt [\{atom(), integer()\}]}, {\tt []}... \pause
\end{itemize}

\begin{itemize}
\item  tuple(): a tuple of any size \pause
\item  list(): a proper list of any type ({\tt [any()]} \pause
\item  list(integer()) : a proper list of integers \pause
\end{itemize} \pause

\end{frame}


\begin{frame}[fragile]{type declarations}

  \pause\vspace{10pt}

  Cards are represented as {\tt \{:card, suit, value\}}, where the
  suit is represented using the atoms {\tt :spade}, {\tt :heart}, {\tt
    :diamond} and {\tt :clubs}.

\vspace{10pt}
How do we specify the type for {\em suit/1}:

\begin{verbatim}
suit({:card, suit, _}) do suit end
\end{verbatim}
\pause\vspace{10pt}
\begin{verbatim}
@spec suit(tuple()) :: atom()
\end{verbatim}

\end{frame}

\begin{frame}[fragile]{defining types}

We would like to define our own type that specifies what a card looks like.

\begin{verbatim}
 @type value() :: 1..13 
\end{verbatim}
\pause

\begin{verbatim}
 @type suit() :: :spade | :heart | :diamond | :clubs 
\end{verbatim}

\pause
\begin{verbatim}
 @type card() :: {:card, suit(), value()}
\end{verbatim}

\pause
\begin{verbatim}
 @spec suit(card()) :: suit() 
\end{verbatim}

\end{frame}

\begin{frame}[fragile]{defining types}

\pause
\begin{verbatim}
 @type boolean() :: true | false
\end{verbatim}

\pause
\begin{verbatim}
 @type byte() :: 1..255 
\end{verbatim}

\pause
\begin{verbatim}
 @type number() :: integer() | float() 
\end{verbatim}
\pause
The type {\tt any()}, defines the union of all types.

\end{frame}


\begin{frame}[fragile]{defining types}

The type list(t) is the type of lists containing elements of type t.
\pause

\begin{verbatim}
 @type list(t) :: [] | [t|list(t)]
\end{verbatim}

\pause
\begin{verbatim}
 @type string() :: list(char())
\end{verbatim}

Define the type of a deck of cards.
\pause
\begin{verbatim}
 @type deck() :: list(card())
\end{verbatim}

\end{frame}


\begin{frame}{program annotation}

Type specifiers are used for:
\begin{itemize}
\item documentation of intended usage
\pause
\item automatic detection of type errors
\end{itemize}

\pause\vspace{10pt}
{\em the compiler does not check types}

\pause\vspace{10pt}
Dialyzer:

\begin{itemize}
\item checks that given specifications agree with call patterns
\item detects exceptions and dead code 
\end{itemize}

\end{frame}


\begin{frame}{dynamically typed}

Elixir is a {\em dynamically typed} language: types are checked and handled at run time.

{\em other dynamically typed languages: PHP, Python, Erlang, Lisp, Prolog}

\vspace{20pt}

Java is a {statically typed} language: types are checked and handled at compile time.

{\em other statically typed languages: C/C++, Haskell, Scala, Rust}

\end{frame}

\begin{frame}[fragile]{statically typed}

  The advantage of a statically typed language:
  \begin{columns}
    \begin{column}{0.5\textwidth}
\begin{verbatim}
typedef struct person {
  int id;
  char name[20];
  char email[20];
} person;

void hello(person *who) {
  printf("Hello %s\n", who->name);
}
\end{verbatim}
    \end{column}
    \begin{column}{0.5\textwidth}
\begin{verbatim}
@type person() :: {:person, 
                       integer(), 
                       binary(), 
                       binary()}
      
def hello({:person, _, name, _}) do
   IO.write("Hello #{name}\n")
end
\end{verbatim}
    \end{column}
  \end{columns}

  \vspace{20pt}
\pause {\em In a statically typed language, the compiled code of 
    {\tt hello()} takes the structure {\tt person} for granted.}
    
\end{frame}

\begin{frame}[fragile]{type inference}

  A statically typed language does not imply that the programmer has
  to specify all types explicitly - the compiler can infer the types (
  Haskell, Rust, ..).

  \vspace{10pt}\pause
\begin{verbatim}
fib 0 = 0
fib 1 = 1
fib n = fib (n-1) + fib (n-2)
\end{verbatim}
  

\end{frame}

\begin{frame}{dynamically vs statically}

The pros and cons of dynamically typed languages:

\begin{itemize}
\item pro: quick to write code
\item pro: compiling an easier task
\item con: induces an overhead at run-time
\item con: errors detected first at run-time (and maybe very late)
\end{itemize}

\pause
{\em So why is Elixir dynamically typed?}

\pause
{\em Easier to handle dynamic code updates in distributed systems.}

\end{frame}

\begin{frame}{Type systems}

\begin{itemize}
\item Elixir is a dynamically typed programming language. \pause
\item External tools (Dialyzer) can check for type errors.  \pause
\item Type specification, if correct, helps in understanding the code.  \pause
\item Dynamically vs statically typed systems - pros and cons.  \pause
\end{itemize}

\end{frame}

\begin{frame}[fragile]{problem}

\begin{verbatim}
 {:car, "Volvo", 
        {:model, "XC60", 2018}, 
        {:engine, "A4", 4, 2000, 140}, 
        {:perf, 4.6, 8.8}}
\end{verbatim}

\vspace{20pt} \pause
  
\begin{verbatim}
def car_brand_model( {:car, brand, {:model, model, _}, _ , _}) do 
  "#{brand} #{model}"
end
\end{verbatim}
\end{frame}


\begin{frame}[fragile]{key-value list}

\begin{verbatim}
 {:car, "Volvo", 
        [{:model, "XC60"},{:year 2018}, {:engine, "A4"}, 
         {:cyl, 4}, {:vol, 2000}, {:power 140}, 
         {:fuel, 4.6}, {:acc 8.8}]}
\end{verbatim}
  
\vspace{20pt} \pause

\begin{verbatim}
def car_brand_model( {:car, brand, prop} ) do
    case List.keyfind(prop, :model, 0) do
      nil -> 
         brand
      {:model, model} ->
         "#{brand} #{model}"
    end
end
\end{verbatim}
\end{frame}

\begin{frame}[fragile]{key-value list}

\vspace{20pt}\pause{\em What is the asymptotic time complexity of {\tt keyfind/3}?}


\vspace{40pt}\pause{\em alternative syntax: {\tt [model: "XC60", year: 2018, ...]}}
  
\end{frame}


\begin{frame}{introducing Maps}

  An efficient implementation of a key-value store with a syntax for pattern matching. 

  \vspace{10pt}\pause

  \begin{itemize}
  \item {\tt \%\{\}} : an empty map \pause
  \item {\tt myCar = \%\{:brand => "Volvo", :model => "XC60", :year= 2008\}} : define properties \pause
  \item {\tt \%\{:model => model\} = myCar} : pattern matching \pause
  \item {\tt newCar = \%\{myCar | :year => 2018\}} : map as template for new map
  \end{itemize}
  

  \vspace{20pt}\pause{\em Still no compiler support to detect errors.}
\end{frame}

\begin{frame}[fragile]{introducing Structs}

\begin{verbatim}
defmodule Car do

  defstruct brand: "",  year: 0,  model: "",  cyl: 0,   power: 0 

  def brand_model(%Car{brand: brand, model: model}) do
    "#{brand} #{model}"
  end

  def year(car = %Car{}) do
    car.year
  end

end
\end{verbatim}

  \vspace{20pt}\pause {\em Requesting a property that is not defined is detected at compile time.}
  
\end{frame}

\begin{frame}{Summary}

  \begin{itemize}
  \item dynamically and statically typed systems: pros and cons
  \item tuples: simple but gives us some problems
  \item key-value lists: what problems do we solve, what remains
  \item Maps: pattern matching and more efficient
  \item Structs: towards the advantage of a statically typed system
  \end{itemize}
  
\end{frame}



\end{document}


