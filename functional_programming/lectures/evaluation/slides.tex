
\usepackage{fancyhdr}
\usepackage{lastpage}
\usepackage[utf8]{inputenc}

% Minted for syntax highliting
\usepackage{minted}
\usemintedstyle{tango}

% Headers/footers styling
\pagestyle{fancy}
\fancyhf{}
\renewcommand{\headrulewidth}{0pt}

% Footer
\lfoot{ID1019}
\cfoot{KTH}
\rfoot{\thepage \hspace{1pt} / \pageref{LastPage}}

%\newcommand{\defaultpagestyle}{\thispagestyle{plain}}
\newcommand{\defaultpagestyle}{\thispagestyle{fancy}}


\title[ID1019 Evaluation]{Evaluation}


\author{Johan Montelius}
\institute{KTH}
\date{\semester}

\begin{document}

\begin{frame}
\titlepage
\end{frame}

\begin{frame}{semantics}

We will define a small subset of the Elixir language and describe the
{\em operational semantics}.


\pause \vspace{40pt}Warning - this is not exactly how Elixir works ... but it could have been.

\end{frame}


\begin{frame}{expressions}

  The language is described using a BNF notation.
  \pause
  \vspace{10pt}
  
\begin{grammar}
<atom> ::= :a | :b | :c | \ldots

<variable> ::= x | y | z | \ldots

<literal> ::= <atom>

<expression> ::= <literal> | <variable> |  '\{' <expression> ',' <expression> '\}'
\end{grammar}

\pause \vspace{20pt} Examples: {\tt \{:a,:b\}} , {\tt \{x,y\}} , {\tt \{:a, \{:b, z\}\}}

\pause \vspace{20pt} Simple expressions are also referred to as {\em terms}.

\end{frame}

\begin{frame}{patterns}

  A {\em pattern} is a syntactical construct that uses almost the same
  syntax as terms.

  \pause
  \vspace{20pt}

\begin{grammar}
<pattern> ::= <literal> 
      \alt <variable> 
      \alt '\_' 
      \alt '\{' <pattern> ',' <pattern> '\}'
\end{grammar}

 \pause \vspace{20pt}
  The \_ symbol can be read as ``don't care''.

\end{frame}


\begin{frame}{sequence}

\begin{grammar}
  <match> ::=  <pattern> '=' <expression>
\end{grammar}

\pause
\begin{grammar}
  <sequence> ::=  <expression> \alt <match> ';' <sequence>
\end{grammar}

\pause\vspace{20pt}

examples:
 \begin{itemize}
   \pause \item {\tt x = :a; \{:b, x\}}
   \pause \item {\tt x = :a; y = \{:b, x\}; \{:a, y\}}
 \end{itemize}

\end{frame}

\begin{frame}{evaluation}

  When we {\em evaluate} sequences, the result will be a structure.
  \pause\vspace{10pt}

$${\rm Atoms} =  \{a,b,c, ...\}$$

$${\rm Structures} = {\rm Atoms} \cup \{\{s_1, s_2\} | s_i \in {\rm Structures}\}$$

\pause \vspace{20pt}
An evaluation can also result in $\perp$, called ``bottoms'', this represents a failed evaluation.

\end{frame}

\begin{frame}{injective mapping}

\pause \vspace{20pt}
For every atom {\tt a}, there is a corresponding structure $s$.

\vspace{10pt}
\hspace{40pt}We write ${\rm a} \mapsto s$.

\pause \vspace{10pt}
\hspace{40pt} {\tt :foo} $\mapsto foo$

\hspace{40pt} {\tt :gurka} $\mapsto gurka$


\pause \vspace{20pt}

{\em For every digit \texttt{1,2,3} (or \texttt{I, II, III}) there is a corresponding number $1,2,3$.}

\pause \vspace{20pt}
{\em Our language could have structures that do not have corresponding terms.}

\end{frame}


\begin{frame}{evaluation}

A sequence is evaluated given an {\em environment}, written $\sigma$ (sigma).

\pause\vspace{20pt}
The environment holds a set of variable substitutions (bindings):
$v/s \in \sigma$, $v$ is a variable and $s$ is a structure.

\pause\vspace{20pt} 
An evaluation of a sequence $e$ given an environment
$\sigma$ is written $E\sigma(e)$. 

\pause\vspace{20pt}
We write:
\vspace{20pt}
$$\frac{{\rm prerequisite}}{E\sigma({\rm expression}) \rightarrow {\rm result}}$$

\vspace{20pt}
where {\em result} is a {\em structure}.

\end{frame}


\begin{frame}{evaluation of expressions}

We have the following rules for evaluation of expressions:

\vspace{10pt}\pause Evaluation of an atom: $$\frac{a \mapsto s}{E\sigma(a) \rightarrow s}$$

\vspace{10pt}\pause Evaluation of a variable: $$\frac{v/s \in \sigma}{E\sigma(v) \rightarrow s}$$

\vspace{10pt}\pause Evaluation of a compund structure: $$\frac{ E\sigma(e_1) \rightarrow s_1 \qquad   E\sigma(e_2) \rightarrow s_2}{E\sigma(\lbrace e_1 , e_2\rbrace) \rightarrow \{s_1, s_2\}}$$
 
\end{frame}

\begin{frame}{evaluation of expressions}

  \vspace{20pt} What if we have $E\sigma(v)$ and $\ v/s \not\in \sigma$? 

  \pause $$\frac{ v/s \not\in \sigma}{E\sigma(v) \rightarrow  \perp}$$

\end{frame}

\begin{frame}{evaluation of expressions}

 assume: $\sigma = \lbrace x/\lbrace a, b\rbrace\rbrace$ 

  \begin{eval}
    \pause $E\sigma({\texttt :c})$ & $\rightarrow $ \pause $c$\\
    \pause $E\sigma({\texttt x})$ & $\rightarrow $ \pause $\lbrace a, b \rbrace$
  \end{eval}

  \vspace{20pt}\pause assume: $\sigma = \lbrace x/a, y/b \rbrace$ 

  \pause \begin{eval}
    $E\sigma({\texttt \{ x, y\}})$  &\pause $\rightarrow \lbrace a , b \rbrace$
  \end{eval}
\end{frame}

\begin{frame}{pattern matching}

  The result of evaluating a {\em pattern matching} is a an extended
  environment. \pause We write: $$P\sigma(p,s) \rightarrow \theta$$
  where $\theta$ (theta) is the extended environment.

\pause\vspace{5pt} Match an atom:
$$\frac{a \mapsto s}{P\sigma(a, s) \rightarrow \sigma}$$ 

\pause\vspace{5pt} Match an unbound variable:
$$\frac{v/t \not\in \sigma}{P\sigma(v, s) \rightarrow \lbrace v/s \rbrace \cup \sigma}$$

\pause\vspace{5pt} Match a bound variable:
$$\frac{v/s \in \sigma}{P\sigma(v, s) \rightarrow \sigma}$$ 

\pause\vspace{5pt} Match {\tt ignore}:
$$\frac{}{P\sigma(\_, s) \rightarrow \sigma}$$ 

\end{frame} 

\begin{frame}{matching failure}

\pause\vspace{20pt} What do we do with $P\sigma(a,s)$ when $a \not\mapsto s$?

\pause\vspace{20pt}

$$\frac{a \not\mapsto s}{P\sigma(a, s) \rightarrow {\rm fail}}$$ 

\pause\vspace{20pt}

$$\frac{
v/t \in \sigma \qquad  t \not\equiv s
}{P\sigma(v, s) \rightarrow {\rm fail}}$$ 


{\em A fail is not the same as $\perp$.}
\end{frame}

\begin{frame}{matching compound strcutures}

If the pattern is a compound pattern, \pause the components of the pattern are matched to their corresponding sub structures.

\pause\vspace{10pt}

$$\frac{P\sigma(p_1, s_1) \rightarrow \sigma'  \quad  P\sigma'(p_2, s_2) \rightarrow \theta}{P\sigma(\lbrace p_1, p_2 \rbrace  , \lbrace s_1, s_2 \rbrace) \rightarrow \theta}$$


\pause \vspace{10pt}
Note that the second part is evaluated in $\sigma'$. 

\pause \vspace{10pt}Example: $P\{\}({\texttt \{x,\{y,x\}\}}, \{a, \{b,c\}\}$)

\vspace{20pt}{\em Match a compund pattern with anyting but a compound structure will fail.}

\end{frame}

\begin{frame}{examples}

assume: $\sigma = \lbrace y/b\rbrace$

\begin{itemize}
  \pause\item $P\sigma({\texttt x} , a) \rightarrow $\pause $\ \lbrace x/a \rbrace  \cup \sigma$
  \pause\item $P\sigma({\texttt y} , b) \rightarrow $\pause $\ \sigma$
  \pause\item $P\sigma({\texttt y} , a) \rightarrow $\pause  $\ \mathrm{fail}$
  \pause\item $P\sigma({\texttt \{ y, y\}} , \lbrace a, b \rbrace) \rightarrow $\pause $\ \mathrm{fail}$
\end{itemize}

\end{frame}

\begin{frame}{pattern matching}

\pause Pattern matching can {\em fail}. 

\pause\vspace{20pt}{\em fail} is different from $\perp$

We will use failing to guide the program execution, more on this later.

\end{frame}

\begin{frame}{evaluation of sequences}

  \pause  A new {\em scope} is created by removing variable bindings from an environmet.

  \vspace{10pt}\pause

$$\frac{\sigma' = \sigma \setminus \lbrace v/t \quad | \quad v/t \in \sigma \quad \wedge \quad  v \quad {\rm in} \quad p\rbrace}{S\sigma(p) \rightarrow \sigma'}$$
  
\vspace{10pt}\pause

A sequence is evaluated one pattern matching expression after the other. 

$$\frac{   
  E\sigma(e) \rightarrow t
  \qquad S\sigma(p) \rightarrow \sigma' 
  \qquad P\sigma'(p, t) \rightarrow \theta
  \qquad E\theta({\rm sequence}) \rightarrow s
}{E\sigma(p = e; {\rm sequence}) \rightarrow s}$$ 

\vfill
{\em Erlang and Elixir differ in how this rule is defined.}
\end{frame}

\begin{frame}{example}

   {\tt x = :a; y = :b; \{x,y\}}

\end{frame}

\begin{frame}{Where are we now}

We have defined the semantics of a programming language (not a
complete language) by defining how expressions are evaluated.

\vspace{20pt} 

\pause Important topics:

\vspace{10pt} 

\begin{itemize}
 \pause \item set of data structures: atoms and compound structures
 \pause \item environment: that binds variables to data structures
 \pause \item expressions: term expressions, pattern matching expressions and sequences
 \pause \item evaluation: from expressions to data structures $E\sigma(e) \rightarrow s$
\end{itemize}

\end{frame}

\begin{frame}{Why}

\vspace{40pt}\hspace{80pt}Why do we do this?

\end{frame}


\begin{frame}{more}
What is missing:
\pause
\begin{itemize}
  \item evaluation of {\em case} (and {\em if} expressions)
  \item evaluation of function applications
\end{itemize}
\end{frame}

\begin{frame}[fragile]{case expression}

\begin{verbatim}
   case  x  do
       :a ->  :foo
       :b ->  :bar
   end
\end{verbatim}

\end{frame}

\begin{frame}{case expression}

\begin{grammar}
     <expression> ::=  <case expression> | ...  

     <case expression> ::= 'case' <expression> 'do' <clauses>  'end' 

     <clauses> ::=   <clause> | <clause> ';' <clauses>

     <clause> ::=  <pattern> '->' <sequence>
\end{grammar}
\end{frame}

\begin{frame}{evaluation of case expression}


$$\frac{E\sigma(e) \rightarrow t \qquad C\sigma(t, {\rm clauses}) \rightarrow s }{E\sigma({\tt case}\ e \ {\tt do}\ {\rm clauses} \ {\tt end}) \rightarrow s}$$
  


 \vspace{20pt}\pause $C\sigma(s, {\rm clauses})$ will select one of
 the clauses based on the patterns of the clauses and then continue the
 evaluation of the sequence of the selected clause.
\end{frame}

\begin{frame}{selection of a clause}

$$\frac{
  S\sigma(p) \rightarrow \sigma' \qquad
  P\sigma'(p, s) \rightarrow \theta \qquad
  \theta \not = {\rm fail} \qquad
  E\theta({\rm sequence}) \rightarrow s}{
C\sigma(s, p \;{\rm ->}\;    {\rm sequence} ; {\rm clauses}) \rightarrow s}$$

\pause \vspace{10pt}

$$\frac{
  S\sigma(p) \rightarrow \sigma'   \qquad
  P\sigma'(p, s) \rightarrow {\rm fail} \qquad
  C\sigma(s, {\rm clauses}) \rightarrow s}{
C\sigma(s, p \;{\rm -\textgreater}\;  {\rm sequence} ; {\rm clauses}) \rightarrow s}$$

\pause \vspace{10pt}

$$\frac{
  S\sigma(p) \rightarrow \sigma' \qquad
  P\sigma'(p, s) \rightarrow {\rm fail} }{
C\sigma(s, p \;{\rm -\textgreater}\;  {\rm sequence}) \rightarrow \perp}$$


\end{frame}

\begin{frame}{example}

\begin{eval}
  \pause$E\lbrace x/\lbrace a,b\rbrace\rbrace(${\tt case x do :a -> :a; \{\_,y\} -> y end}$) \rightarrow$ 
\end{eval}

\begin{eval}
   \hspace{40pt}\pause$E\lbrace X/\lbrace a,b\rbrace\rbrace(${\tt x}$) \rightarrow \lbrace a,b\rbrace$
\end{eval}

\begin{eval}
   \pause $C\lbrace X/\lbrace a,b\rbrace\rbrace(\lbrace a,b\rbrace, ${\tt \ :a -> :a; \{\_,y\} -> y}$) \rightarrow$ 
\end{eval}
\begin{eval}
   \hspace{40pt}\pause $P\lbrace x/\lbrace a,b\rbrace\rbrace\rbrace( ${\tt :a}$, \lbrace a,b\rbrace) \rightarrow \mathrm{fail}$
\end{eval}

\begin{eval}
   \pause $C\lbrace x/\lbrace a,b\rbrace\rbrace(\lbrace a,b\rbrace, ${\tt \  \{\_,y\} -> y}$) \rightarrow$ 
\end{eval}
\begin{eval}
   \hspace{40pt}\pause $P\lbrace x/\lbrace a,b\rbrace\rbrace(${\tt \{\_,y\}}$, \lbrace a,b\rbrace) \rightarrow $ \pause $\lbrace y/b,\  x/\lbrace a,b\rbrace\rbrace$
\end{eval}

\begin{eval}
  \pause$E\lbrace y/b, x/\lbrace a,b\rbrace\rbrace(${\tt y}$) \rightarrow $
\end{eval}
\begin{eval}
  \hspace{40pt}\pause$b$
\end{eval}
  
\end{frame}


\begin{frame}{free variables}

Are all syntactical correct sequences also valid sequences?

\pause\vspace{20pt}

A sequence must not contain any {\em free variables}.

\pause\vspace{10pt}

A free variable in a {\tt <sequence>} is bound by the
pattern matching expressions in the sequence {\tt <patter> =
<expression>, <sequence>} if the variable occurs in the {\tt
<pattern>}.

\pause\vspace{10pt}

A free variable in a {\tt <sequence>} is bound by the pattern matching
expressions in the clause {\tt <pattern> -> <sequence>} if the
variable occurs in the {\tt <pattern>}.


\end{frame}

\begin{frame}{free variables}

{\tt x = :a; \{y,z\} = \{x,:b\}; \{x,y,z\}}

\pause\vspace{20pt}

{\tt \{y,z\} = \{x,:b\}; \{x,y,z\}}

\pause\vspace{20pt}

{\tt x = \{:a,:b\}; case x do \{:a,z\} -> z end}

\end{frame}

\begin{frame}{much ado about nothing}

A lot of work for something that simple - why bother, it could not have
been done differently.
\end{frame}


\begin{frame}[fragile]{variable scope}

\vspace{20pt}

\hspace{20pt}
\begin{verbatim}
   x = {:a,:b};
   y = case x do
         {:a, z} -> {:c, z}
       end;
   {y, z}
\end{verbatim}

\vspace{20pt}{\em This is not allowed in our language, \texttt{z} in
  \texttt{\{y,z\}} is a free variable. However .... \pause is allowed
  in Erlang and was until changed allowed in Elixir (fixed in v1.5).}

\end{frame}
 
\begin{frame}{what's missing}

\vspace{20pt}\hspace{20pt} Handle lambda expressions, closures and function application.
  
\end{frame}




\end{document}
